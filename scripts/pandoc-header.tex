\usepackage{float}
\usepackage[section]{placeins}
\usepackage{datetime2}

% Kapitel-Überschrift ohne "Chapter 1" Zeile
\usepackage{titlesec}

\titleformat{\chapter}[display]{\normalfont\huge\bfseries}{}{0pt}{}

% Logo einbinden
\usepackage{graphicx}

% Header/Footer
\usepackage{fancyhdr}
\usepackage{lastpage}

\pagestyle{fancy}
\fancyhf{} % alles löschen

% Platz für Header (wichtig bei Logos)
\setlength{\headheight}{32pt}
\setlength{\headsep}{12pt}

% Optional: Linie im Header
\renewcommand{\headrulewidth}{0.4pt}

% ----- HEADER -----
% Links: Projektname
%\lhead{\textbf{myApp-final}}
% Nur den Kapitel-Titel in \leftmark schreiben (ohne "Chapter 1.")
\renewcommand{\chaptermark}[1]{\markboth{#1}{}}
% Mitte: Kapitelname (nicht uppercase)
\chead{\nouppercase{\leftmark}}
% Rechts: Logo (vertikal sauber ausgerichtet)
\rhead{\raisebox{-0.2\height}{\includegraphics[height=34pt]{scripts/logo.png}}}

% ----- FOOTER -----


% Footer
\fancyfoot[L]{Created: \DTMnow}
% Seite X / Y (alle Seiten)
\fancyfoot[R]{\thepage\ /\ \pageref{LastPage}}

% Erste Seite: komplett leer (aber sie zählt mit)
\AtBeginDocument{\thispagestyle{empty}}

% Optional: sicherstellen, dass Kapitel/TOC-Seiten nicht wieder auf "plain" zurückfallen
\makeatletter
\let\ps@plain\ps@fancy
\makeatother
